\documentclass{article}
\usepackage[UTF8]{ctex}
\usepackage{geometry}
\usepackage{multirow}
\usepackage{natbib}
\geometry{left=3.18cm,right=3.18cm,top=2.54cm,bottom=2.54cm}
\usepackage{graphicx}
\pagestyle{plain}	
\usepackage{setspace}
\usepackage{enumerate}
\usepackage{caption2}
\usepackage{float}
\usepackage{datetime} %日期
\renewcommand{\today}{\number\year 年 \number\month 月 \number\day 日}
\renewcommand{\captionlabelfont}{\small}
\renewcommand{\captionfont}{\small}
\begin{document}

\begin{figure}
    \centering
    \includegraphics[width=8cm]{upc.png}

    \label{figupc}
\end{figure}

	\begin{center}
		\quad \\
		\quad \\
		\heiti \fontsize{45}{17} \quad \quad \quad 
		\vskip 1.5cm
		\heiti \zihao{2} 《计算科学导论》个人职业规划
	\end{center}
	\vskip 2.0cm
		
	\begin{quotation}
% 	\begin{center}
		\doublespacing
		
        \zihao{4}\par\setlength\parindent{7em}
		\quad 

		学生姓名:\underline{\qquad  张永贞 \qquad \qquad}

		学\hspace{0.61cm} 号:\underline{\qquad 2007020104\qquad}
		
		专业班级:\underline{\qquad 本研2001 \qquad  }
		
        学\hspace{0.61cm} 院:\underline{计算机科学与技术学院}
% 	\end{center}
		\vskip 1.5cm
		\centering
		\begin{table}[h]
            \centering 
            \zihao{4}
            \begin{tabular}{|c|c|c|c|c|c|c|c|c|}
            % 这里的rl 与表格对应可以看到,姓名是r,右对齐的;学号是l,左对齐的;若想居中,使用c关键字。
                \hline
                \multicolumn{5}{|c|}{分项评价} &\multicolumn{2}{c|}{整体评价}  & 总    分 & 评 阅 教 师\\
                \hline
                自我 & 环境 & 职业 & 实施 & 评估与 & 完整性 & 可行性 &\multirow{2}*{} &\multirow{2}*{}\\
                分析& 分析& 定位 & 方案 & 调整 & 20\% & 20\% & ~&~ \\\            
                10\% & 10\% & 15\% & 15\% & 10\% & &  &~ &~\\
                \cline{1-7} 
                & & & & & & & ~&~ \\
                & & & & & & & ~&~ \\
                \hline      
            \end{tabular}
        \end{table}
		\vskip 2cm
		\today
	\end{quotation}

\thispagestyle{empty}
\newpage
\setcounter{page}{1}
% 在这之前是封面,在这之后是正文
\section{自我分析}
	自我分析即对自己进行全方位、多角度的分析,目的是认识自己、了解自己。只有认识了自己,才能对自己的职业做出正确的选择,才能选定适合自己发展的职业生涯路线,才能对自己的职业生涯目标做出最佳抉择。\par
\subsection{自然条件}
性别女、年龄十九、身体条件良好、健康状况良好、居住于潍坊市的一个农村。\par
\subsection{性格分析}
自己最初是个特别内向的女孩,没什么主见,很不喜欢与他人交流,受了委屈就喜欢掉眼泪。随着岁月的增长,父母能为自己拿主意的时候变得越来越少,时代也在不断的给自己提出更高的要求,自己也知道不能一直这样,必须从自身做出改变。因此,自己也做出了很多尝试:之前除非硬性要求从来不在群里说话,因为觉得万一说话没人接,或者消息刷的太快,没人理自己会很尴尬。为了做出改变,还没来校之前就尝试着在迎新群与其他人聊天,前几次心里确实还是战战兢兢,把自己的消息发出去后捧着手机,忐忑地等待回音,幸亏学长学姐很热情地解答了自己的问题,在群里聊天多了之后就会感到,就算没人理你也不会有什么大不了,除此之外,还认识了许多新朋友,那种感觉确实很让人高兴。另外一件事是新生游园时的自我介绍,可能我们班的人都很羞涩,没有一个人主动上,弄得带我们的学长也很心累,最后他没辙了,问我能不能先来,毕竟他也就因为我在群里比较活跃对我有点印象,当时心里可紧张,尤其自己除了姓名家乡外也没什么可以拿得出手介绍的东西,又觉得自己说个姓名家乡就下去太不给学长面子,一紧张就说了个自己喜欢看耽美小说,然后就下去了。虽然我自己觉得可尴尬,但是别人竟然觉得我说的挺好,当天就有同学加我QQ,说我落落大方,也是挺惊讶的了。还有本研班面试的时候,面试之前捏着自己写的稿子特别紧张,给自己心理建设半天,等真正进房间面试之后虽然紧张的一张脸也没记住,但起码外表上看不出来,记得当时还有个女老师说这些孩子都不紧张,其实我很紧张就是没显在面上。虽然现在有些时候自己可能还是会畏缩不前,但相比之前已经进步了很多,相信自己会越来越好。\par
\subsection{教育与学习经历}
小学毕业于河沟英才小学,初中毕业于侯镇二中,高中毕业于寿光中学,目前就读于中国石油大学(华东)。\par
\subsection{工作与社会阅历}
在过去的十九年除了学习并没有参加什么活动,因此在这一方面没有太多话可说。\par
\subsection{知识、技能与经验}
虽然完整地走完了九年义务教育和三年高中生活,但除此之外自己的生活乏善可陈,在学校的鼓励下虽参加过多次比赛,但规模不大,含金量也不高,能获得名次的大多是校级和县级比赛,可能拿到的最高荣誉是山东省的语文报杯作文比赛的一等奖,但这个比赛的知名度并不高,因此也不值得自己拿出来夸夸其谈。\par
\subsection{兴趣爱好与特长}
爱好阅读,但并不喜欢那些晦涩难懂又总不说明话的煌煌巨著,也啃不动那些全是专业名词的课本和工具书,只是喜欢轻松地读些言情和纯爱文学,虽然并不为众人所称颂,但确实是我感到最轻松愉悦的事情。没有什么专精的特长,非要说的话,得益于多年阅读,对写作略有心得吧。\par
\section{环境分析}
环境分析主要是评估周边各种环境因素对自己职业生涯发展的影响。每一个人都处在一定的环境之中,职业发展必然要受到所处环境的影响,只有充分了解和把握所处环境的现状、特点、发展变化趋势,才能做到在复杂的环境中避害趋利,使你的职业生涯规划具有实际意义。\par
\subsection{社会环境分析}
政治形势:习近平时代从2012年开始,预计将持续到2032年,世界格局逐渐从中美争霸转向东西之争。习总书记上台后反腐工作轰轰烈烈,因此政治比较清明,政局比较稳定,中国走在行稳致远的路上。\par
经济形势:虽然中美贸易摩擦不断,但是,“中国经济是一片大海,而不是一个小池塘;狂风骤雨可以掀翻小池塘,但不能掀翻大海;经历了无数次狂风骤雨,大海依旧在那儿!”另外,我国经济已经进入工业化晚期,城乡水平迅速拉平。IT产业是世界第一大产业,未来IT产业将改变世界所有产业。\par
就业形势:从疫情后复工复产情况来看,中国是最早实现经济正增长的国家,相较于其他国家的惨淡形势,中国的就业形势还是一片光明的。而计算机行业前景大好,仍具备着充沛的活力。\par
\subsection{家庭环境分析}
婚姻状况:家庭美满,生活幸福。\par
经济状况:目前暂无经济能力,仍缺少不了父母的帮助。家庭属于工薪家庭,未来需要自己来分担家庭的负担。\par
家人期望:最初选择专业时,父母老师都是认为女孩子还是更适合老师、会计之类的工作,因此推荐我选择师范这类专业,我最终的选择可能并不符合他们的心理预期,但我的家庭一直是个开明的家庭,他们只是给出自己的建议,而非将自己的想法强加在我的身上,我认为这是我的一种幸运。\par
家族传统:亲人都是农民或工人,不曾涉足计算机行业。\par
\subsection{职业环境分析}
行业现状及发展趋势:\par
计算机行业一直以来人们印象中的高薪行业,人工智能前些年也是炙手可热,但是计算机行业真正获得巨大利润的仍然是行业内的前百分之十,人工智能的第三次浪潮也已趋近尾声,更多的程序员仍是996的社畜,任何行业都没有优劣,有的只是从业人员的水平高低。\par
职业的工作内容、工作要求:\par
1、对项目经理负责,负责软件项目的详细设计、编码和内部测试的组织实施,对程序员小型软件项目兼任系统分析工作,完成分配项目的实施和技术支持工作。\par
2、协助项目经理和相关人员同客户进行沟通,保持良好的客户关系。\par
3、参与需求调研、项目可行性分析、技术可行性分析和需求分析。\par
4、熟悉并熟练掌握交付软件部开发的软件项目的相关软件技术。\par
5、负责向项目经理及时反馈软件开发中的情况,并根据实际情况提出改进建议。\par
6、参与软件开发和维护过程中重大技术问题的解决,参与软件首次安装调试、数据割接、用户培训和项目推广。\par
7、负责相关技术文档的拟订。\par
8、负责对业务领域内的技术发展动态进行分析研究。\par


\subsection{地域与人际环境分析}
毕业之后可能没有足够的能力和心理承受力去北京等工作压力大的城市竞争,自己的大学是在青岛读的,离家也不算远,因此预计会留在青岛工作。\par
工作城市的气候水土:南北方向上地处中国中部,没有东北的严寒,也没有南方的炎热多雨,是自己最适应的气候。\par
文化特点:青岛历史文化悠久,1994年被国务院公布为国家历史文化名城。\par
发展前景:青岛是仅次于北京和天津的北方的第三大城市,是少数一直在向前发展的城市,发展前景大好。\par
人脉与人际关系:因为大学在青岛,相信不少同学也会留在青岛工作,此外,与自己关系亲近的老师也能为自己提供帮助,大学期间积累下的人脉将是自己人生中非常宝贵的一笔财富。\par
\par 

\section{职业定位}
在准确地对自己和环境做出了分析之后,确定适合自己行业和有实现可能的职业发展目标。职业定位时要注意与自己的自然条件、知识背景、技能特长、性格特点、兴趣爱好是否匹配,考虑与自己所处的环境是否相适应。职业定位决定了职业发展中的行为和结果,是制定职业生涯规划的关键,应当科学合理,具有可行性。\par

\begin{figure}[H]
\centering
\includegraphics[scale=0.2]{工作环境}
\label{fig:universe}
\end{figure}
\subsection{行业领域定位与理由}
既然选择了这个专业,未来就最有可能成为一名程序员,一是专业对口,二是自己虽然已经慢慢提高自己的交流能力,但内心却并不热衷于此,相较之下,还是更喜欢在自己的领域里默默做好自己的事,因此,程序员还是适合我的。\par
\subsection{职业岗位起点定位与理由}
从最基本的程序员做起,毕竟在进入大学之前,自己从来没有接触过计算机方面的知识,就大一现在在学的专业课C++来说,编程能力一塌糊涂,数学能力也有待提升,实在不敢托大。\par
\subsection{职业目标与可行性分析}
成果目标:做出自己的东西\par
经济目标:年薪20万\par
能力目标:熟练掌握编程语言、数据库和操作系统等\par
职务目标:系统分析师\par

\begin{enumerate}[(1)]
	\item 短期目标(大学4年)\par
	 大一熟练掌握C++、python,学好数学分析、离散数学、概率论与数理统计、大学物理等,打好理科基础,平常多加练习英语,提高英语阅读能力和口语水平;\par
     大二熟练掌握Java、数据结构与算法、数字逻辑电路、数据库原理、机器学习,学好线性代数、数值计算方法、数学建模;\par
     大三学好可视化、深度学习、云计算、自然语言处理等等;\par
     大四学好算法设计与分析、虚拟现实、最优化等等。\par
	\item 中长期目标(5-10年)。\par
	顺利进入硕士研究生阶段,在这一阶段在研究上有所突破和创新,有可能的话,希望可以读博士。\par

\end{enumerate}

\section{实施方案}
在明确了职业定位后,要制定实现职业生涯目标的行动方案,不付诸行动,职业目标只能是一种梦想。实施方案是实现职业目标的保证,尽量考虑周全、具有可操作性。\par

\begin{enumerate}[1、]
	\item 如何利用现有条件和自身优势以实现职业生涯目标。\par
	目前自己正处于大一阶段,最重要的便是打好基础,记得老师曾说过,我们最需要的是三种能力,一是数学的能力,二是英语的能力,三是编程的能力,而这三项都是从大一开始便有课程的,千万不能落下,要做到自律,尽量减少娱乐时间,将更多的时间放在学习上。\par
	\item 如何克服缺点、弥补不足、增长知识、提高能力以实现职业生涯目标。\par
	虽然自己高考时选择了物理、化学和生物三科,算得上是山东省高考改革之前的理科生,但实际自己的理科并不强,数学可以说是弱科,数学高考成绩也相当差劲,本研班又学的是比高等数学更难的工科数学分析,这个学期的数学学习也并不理想,除此之外,英语、编程也差强人意,在即将到来的较长的寒假里,计划回家过年的同时加强对弱科的学习,以期提高自己的能力。另外,主动交流,多参加讲座报告等,像海绵一样吸收周围的知识,可能会得到很多意想不到的收获,例如,有一次报名了一个讲座,最初的目的只是水学时,但讲座过程中,成功嫖到了科学上网的方法。\par
	\item 如何处理人际关系和发展人脉以实现职业生涯目标。\par
	多参加文体活动、讲座报告、创新竞赛和社会实践等,可以遇到志同道合的人,向优秀的人汲取经验;遇到问题主动向学长学姐请教,获得知识的同时发展了人脉;有任务积极主动承担,学会推销自己,尽快让他人对自己留下印象,这一点王芳老师曾多次提到,主动竞选班委、节日主动表达祝福、多去找老师聊天都可以推销自己。\par
	\item 如何处理工作与家庭、生活的关系以实现职业生涯目标。\par
	我的家庭是个开明的家庭,父母没有过多干预自己的选择,只要平心静气地好好谈,家庭这方面不会有太大问题。\par
	\item 如何处理释放工作压力、保证身心健康以实现职业生涯目标。\par
	虽然一直说要努力学习,但也不必将生活过得太过压抑,压力大的时候可以四处走走,放松下心情,偶尔打会儿游戏、看会儿小说也无可厚非,只要不过度,它们反而是有利于提高自己的学习效率的。\par

\end{enumerate}

\section{评估与调整}
由于影响职业生涯规划的因素很多,且大都处于动态变化之中,因此职业生涯规划应定期评估,并根据影响因素的变化和实施结果的情况及时作出调整,这样才能保证其行之有效。\par 
\subsection{评估时间}
每学期评估一次。\par
\subsection{评估内容}
每学期末评估自己课程学习是否达到要求,如果达到了,在达到过程中记录可借鉴的经验,并运用到以后的学习当中;如果没有达到,分析未达到的原因,在进一步的学习中尽量避免,并制定计划对未完成的内容进行弥补。\par
另外可以从成果目标、经济目标、能力目标、职务目标等方面总结本科、研究生阶段的大目标,确定哪些目标已按预期实现,哪些目标尚未达到,对已实现的成果总结经验,对未完成的目标分析原因。\par
\subsection{调整原则}
当多次未完成目标时,应重新考虑目标是否符合自身情况,并对目标酌情修改。\par
倘若目标实现过于简单,有过多的空余时间,也应重新考虑能否对自己提高要求。\par
目标更改不能跨度过大,应逐级修改。\par
当遇到自己难以决定的问题时,可以寻求老师的帮助。\par




\end{document}
