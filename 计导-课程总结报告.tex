\documentclass{article}
\usepackage[UTF8]{ctex}
\usepackage{geometry}
\usepackage{natbib}
\geometry{left=3.18cm,right=3.18cm,top=2.54cm,bottom=2.54cm}
\usepackage{graphicx}
\pagestyle{plain}	
\usepackage{setspace}
\usepackage{caption2}
\usepackage{datetime} %日期
\usepackage{float}
\renewcommand{\today}{\number\year 年 \number\month 月 \number\day 日}
\renewcommand{\captionlabelfont}{\small}
\renewcommand{\captionfont}{\small}
\bibliographystyle{unsrt}
\begin{document}

\begin{figure}
    \centering
    \includegraphics[width=8cm]{upc.png}

    \label{figupc}
\end{figure}

	\begin{center}
		\quad \\
		\quad \\
		\heiti \fontsize{45}{17} \quad \quad \quad 
		\vskip 1.5cm
		\heiti \zihao{2} 《计算科学导论》课程总结报告
	\end{center}
	\vskip 2.0cm
		
	\begin{quotation}
% 	\begin{center}
		\doublespacing
		
        \zihao{4}\par\setlength\parindent{7em}
		\quad 

		学生姓名:\underline{\qquad  张永贞 \qquad \qquad}

		学\hspace{0.61cm} 号:\underline{\qquad 2007020104\qquad}
		
		专业班级:\underline{\qquad 本研2001 \qquad  }
		
        学\hspace{0.61cm} 院:\underline{计算机科学与技术学院}
% 	\end{center}
		\vskip 2cm
		\centering
		\begin{table}[h]
            \centering 
            \zihao{4}
            \begin{tabular}{|c|c|c|c|c|c|c|}
            % 这里的rl 与表格对应可以看到,姓名是r,右对齐的;学号是l,左对齐的;若想居中,使用c关键字。
                \hline
                课程认识 & 问题思 考 & 格式规范  & IT工具  & Latex附加  & 总分 & 评阅教师 \\
                30\% & 30\% & 20\% & 20\% & 10\% &  &  \\
                \hline
                 & & & & & &\\
                & & & & & &\\
                \hline
            \end{tabular}
        \end{table}
		\vskip 2cm
		\today
	\end{quotation}

\thispagestyle{empty}
\newpage
\setcounter{page}{1}
% 在这之前是封面,在这之后是正文
\section{引言}
导论是每个专业第一学期必修的一门课程,《计算科学导论》作为计算机专业的导论课,为尚不了解计算机的我们揭开了它神秘的面纱,为我们将来的发展指明了方向。

\section{对计算科学导论这门课程的认识、体会}
《计算科学导论》作为一门导引性课程,毋庸置疑为我们勾勒起整个计算机领域的框架,从更一般的认识层面上掌握学习知识的方式方法,带我们了解计算机的发展历史,给我们讲述计算机的相关知识,让我们更好地认识计算机、学习计算机。\par
计算科学与计算机科学不同,计算机科学是人们人为地将计算机科学和计算机工程割裂开来,而实际上,计算机科学与计算机工程是密不可分的,没有计算机科学,计算机工程就如同空中楼阁,转眼倾塌;没有计算机工程,计算机科学空有一身理论却无用武之地,因此,用计算科学一词涵盖计算这一领域内的所有工作。\par
计算科学意义重大,它是继理论科学和实验科学的第三种学科类形态,它可以通过大量的数据学习来模拟现实,开辟了一种全新的从事一类学科研究与发展的文化方式,有利于人类以新的方式开展工作。\par
\subsection{逻辑与人工智能}
人工智能(英语:artificial intelligence,缩写为AI)亦称智械、机器智能,指由人制造出来的机器所表现出来的智能。通常人工智能是指通过普通计算机程序来呈现人类智能的技术。该词也指出研究这样的智能系统是否能够实现,以及如何实现。同时,通过医学、神经科学、机器人学及统计学等的进步,常态预测则认为人类的很多职业也逐渐被其取代。 \citep{a1}\par
这是人们常规思想下对人工智能的认识,也是不久之前我的认知。直到选择了这个专业,接触了这方面的知识,才了解到一定程度上人工智能跟人工智障是同义词,人工智能远没有人们想象中那么强大,虽然感知智能已经初具雏形,目前的图像识别、语音识别等方面也做得较为成功,但在认知智能方面,目前只能代替人脑中部分机械的脑力劳动,实际操作中完全达不到人们想象中那样神奇,许多职业将被代替从而导致很多人失业更是无稽之谈。记得新生研讨课上,研究智慧医疗的王老师曾提到过,在她们与医院方面交涉的过程中,遇到了很大的阻力,医生们对她们的到来很不客气,因为他们认为智慧医疗会取代他们的工作,让他们失业。但事实却并非如此,人工智能的发展远没有人们心中想象的那么完美。作为人工智能方面的学生,我们更应该认清形势,清晰地认识到人工智能的优点和不足,把握好自己的未来和方向。\par
老师在课上提到了人工智能的三次浪潮,为此我查阅了部分相关资料:\par
\begin{figure}[H]
\centering
\includegraphics[scale=0.2]{三次浪潮}
\label{fig:universe}
\end{figure}
人工智能起源于1956年美国达特茅斯学院举办的夏季学术研讨会。在这次会议上, 达特茅斯学院助理教授John Mc Carthy提出的“人工智能 (Artificial Intelligence, AI) ”这一术语首次正式使用。之后, 人工智能的先驱艾伦·图灵提出了著名的“图灵测试”:在人机分隔的情况下进行测试, 如果有超过百分之三十的测试者不能确定被试是人还是机器, 那么这台机器就通过了测试, 并被认为具有人工智能。图灵测试掀起了人工智能的第一轮浪潮。在人工智能研究方法上, 以抽象符号为基础, 基于逻辑推理的符号主义方法盛行, 其突出表现为:在人机交互过程中数学证明、知识推理和专家系统等形式化方法的应用。但在电子计算机诞生的早期, 有限的运算速度严重制约了人工智能的发展。 \par
20世纪80年代, 人工智能再次兴起。传统的符号主义学派发展缓慢, 有研究者大胆尝试基于概率统计模型的新方法, 语音识别、机器翻译取得了明显进展, 人工神经网络在模式识别等领域初露端倪。但这一时期的人工智能受限于数据量与测试环境, 尚处于学术研究和实验室中, 不具备普遍意义上的实用价值。\par
人工智能的第三次浪潮缘起于2006年Hinton等人提出的深度学习技术。Image Net竞赛代表了计算机智能图像识别领域最前沿的发展水平, 2015年基于深度学习的人工智能算法在图像识别准确率方面第一次超越了人类肉眼, 人工智能实现了飞跃性的发展。随着机器视觉研究的突破, 深度学习在语音识别、数据挖掘、自然语言处理等不同研究领域相继取得突破性进展。2016年, 微软将英语语音识别词错率降低至百分之五点九, 可与人类相媲美。如今, 人工智能已由实验室走向市场, 无人驾驶、智能助理、新闻推荐与撰稿、搜索引擎、机器人等应用已经走进社会生活。因此, 2017年也被称为人工智能产业化元年。\citep{a2}\par
人工智能第一次浪潮时间从1956年到1976年持续了20年的时间,第二次浪潮从1976年到2006年持续了30年的时间,而第三次浪潮开始于2006年,姑且将人工智能浪潮周期看作二十年,也就是说当2026年本研班毕业的那一年,人工智能的第三次浪潮将接近结束,人工智能的研究可能会跌入低谷,这对我们来说,这或许并不是一个好消息,但把握好时代的发展脉络,仍有利于我们做出最好的选择。\par

\subsection{五眼联盟}
五眼(英语:Five Eyes,缩写为 FVEY),又译为五眼联盟,是由五个英语圈国家所组成的情报联盟,在英美协定下组成的国际情报分享团体,成员国包括澳大利亚、加拿大、新西兰、英国和美国。五眼联盟的历史最早可以追溯到第二次世界大战,同盟国所发布的大西洋宪章。冷战期间,ECHELON监视系统被五眼联盟所开发,用于监视苏联及其东欧盟友。\citep{a3}\citep{a4}\citep{a5}\par
\begin{table}[h]
    \centering
\begin{tabular}{rl}
% 这里的rl 与表格对应可以看到,姓名是r,右对齐的;学号是l,左对齐的;若想居中,使用c关键字。
    \hline
    国家 & 机构 \\
    \hline
    美国 & CIA DIA FBI NGA NSA\\ 
    英国 & DI GCHQ MI5 MI6 \\
    加拿大 & CDI CSE CSIS\\
    澳大利亚 & ASIS ASD ASIO AGO DIO \\
    新西兰 & DDIS GCSB NZSIS \\
    \hline
\end{tabular}
    \label{table1}
\end{table}
五眼联盟对其他国家的政府、企业进行窃听和监视,并将信息在五眼联盟内部共享,严重违反了国际法和国际关系准则,应该受到全人类的抵制。但是,即使五眼联盟是这样臭名昭著,仍有其他国家譬如日本企图加入他们,成为五眼联盟的“第六只眼”。此外,美国封杀华为也与五眼联盟有莫大的关系,因为更多的人使用华为而非美国系统,阻碍了美国利用海底电缆监控世界的脚步,反对五眼联盟应成为全人类的共识。\par

\section{分组演讲的进一步的思考}
\begin{figure}[H]
\centering
\includegraphics[scale=0.2]{演讲}
\label{fig:universe}
\end{figure}
我们组选择的主题是无人机。\par
无人机,顾名思义,是无人驾驶的飞机,它是利用无线电遥控设备和自备的程序控制装置操纵的不载人飞机,简单的来说,它是指可回收、能遥控操作和自主飞行的无人驾驶飞机。\par
我们说,“伟大的科技都来自军事”,无人机也不例外。无人机的历史可以追溯到1914年,当时,第一次世界大战正进行得如火如荼,英国的两位将军卡德尔和皮切尔提议,能不能发明一种无人机器,让它可以飞到敌军上空,投掷炸弹。这个想法一经提出就得到了英国军事航空学会的理事长戴·亨德森的赏识,于是他派人加紧研制无人机。但结果并不尽如人意,1917年,英国无人机两次试飞失败,而与此同时,美国人后来居上,库柏和斯佩里发明了自动陀螺稳定仪,这种仪器可以使无人机保持稳定,向前飞行,在此基础上发明了“空中鱼雷”式无人机。此后,各式各样的无人机层出不穷,但是,它们飞出后都不能飞回原点,也就是不具有“可回收”的特点,因此都算不上真正的无人机。直到1935年,英国的“蜂王”式无人机问世,这是无人机真正开始的时代,“蜂王”式无人机也被称为无人机的开山鼻祖。但是,当时的科技水平不高,无人机的各项性能满足不了实战的需求,在战争中没有发挥应有的作用,因此,无人机备受冷落,美国国防部甚至在七十年代终止了无人机的发展计划。直到1982年,以色列首创无人机与有人机协同作战,在贝卡谷地战争中重创叙利亚,无人机重回大家的视野,九十年代,美国国防部拨款支持无人机的研发事业。21世纪初,无人机的体型由大转小,出现了迷你无人机,同时,无人机由军用过渡到民用,2006年,影响世界民用无人机的大疆无人机公司成立。\par
此后,无人机一路发展,发展到现在,大致可以分为两类,一类是军用,一类是民用,民用无人机又可以分为消费级无人机和工业级无人机,消费级无人机就是用于消费、娱乐的无人机,可以简单的理解为航拍无人机,而工业级无人机是指针对某一个领域展开应用,例如农业植保、物流配送、警用消防等。目前,消费级无人机的市场已趋近饱和,因此,未来无人机的发展主要将在军用无人机和工业级无人机展开。\par
军用无人机不必多说,在未来的信息化战争中,这是每个国家都会重点关注的一个领域。工业级无人机有很多应用领域,在物流配送方面,偏远地区的物流量较小且交通不便,而无人机配送不仅可以解决偏远地区的配送问题,而且节约了人力成本,提高了效率。另外,疫情期间,无人机配送还可以降低人们面对面交接货物时带来的传播疫情的风险。在农林植保方面,人工喷洒农药不仅工作量大,农药中的有害物质还会危害农民的身体,而且在田地里走动还会对作物造成物理伤害,而无人机喷洒农药就可以很好地避免这些问题。在数据采集方面,无人机的航拍功能让它在气象、交通、自然灾害等方面发挥了很大的作用,例如,春节期间,无人机可以拍摄路口的道路状况,人们可以根据车流量合理规划出行路线。\par
但是,目前这些应用还存在不少问题,例如配送方面的安全问题和农业方面单机作业的低效率问题,为此,人们在很多方面做出了努力。\par
自主导航和避障方面,今年八月份,提出了一种非线性模型预测控制计算技术(Nonlinear Model Predictive Control),简称NMPC,实验者抛出障碍物,而无人机利用NMPC算法模拟出障碍物的运行轨迹,从而达到躲避障碍物的目的。续航方面,目前无人机的续航时间在20分钟到30分钟之间,充电时间长、飞行时间短是无人机的一个致命短板,为此,人们在很多方向做出了尝试,一个方向是太阳能充电,2019年7月 西北工业大学的魅影太阳能无人机实现27小时37分的续航记录;另一个方向是激光充电,中国的短距离激光充能技术成功将续航提升至48个小时,这在世界上都是处于领先地位的。通信技术方面,无人机的数据传输能力称为数传,图像传输能力称为图传,目前,4G的图传能力较弱,分辨率较低,图像不够清晰,在查看设备指示灯和人脸识别等方面还是不能满足用户的需求,但是,我们可以将现在的5G与无人机联系起来,可以极大地提高分辨率和清晰度。组网方面,上面有提到,农业方面单机作业效率仍然比较低,另外,有些地方受制于地形因素限制无人机和地面控制站的通信等诸方面的限制,无人机还无法到达作业区域,影响任务展开区域。无人机自组网技术就是将多架无人机与地面控制站联系起来,形成一个无人机自组网系统,每架无人机都拥有任务节点和中继节点的作用,也就是说,它既可以直接接收、执行来自地面控制站的任务,而且可以发挥路由功能,选择合适的传输路径进行数据传输。多无人机相互协作完成对任务环境的感知,通过自组网技术实现多无人机之间的信息的快速传递共享,完成对任务区域的大范围监测当视距通信陷入盲区时候,可以通过多机中继实现无死角覆盖无人机群系统不依赖于单独的个体,当部分个体离开或加入群系统时,整个群系统仍具有一定的完整性,可继续执行任务等。\par
虽然目前无人机还存在很多问题,但我们相信,随着技术的发展,无人机前景可待,未来可期。
\begin{itemize}
    \item 大疆无人机公司为什么会在2006年崛起,而不是1996年或者其他时间?\par
    1.锂电池技术的逐步成熟\par
    军用无人机之所以能够长时间地在天空飞行,是因为它自身携带着发电机、内燃机、油箱等,不需要电池来供电,自然也不必担心电量问题,无人机既然要由军用过渡到民用,自然是重量越轻越好,势必要减小自己的体型,但这当中存在着一个矛盾,要减少重量,就不可避免地要降低电池的容量,从而导致续航时间减少,这是民用无人机发展的一个短板。此时,锂电池技术的成熟为这一问题找到了解决方案。锂的比能量较高,也就是说锂电池可以以更低的重量储存更多的电量,完美解决了无人机轻巧有余而电量不足的问题。\par
     2.通信技术的进一步发展\par
    民用无人机中的消费级无人机很重要的一个功能是航拍功能,要得到高空角度下较为清晰的照片,这对无人机的图传能力提出了较高的要求。而大疆作为3G浪潮的弄潮儿,站在了时代的风口之上。发展到现在5G技术,大疆的4K无人机传图、传视频已经是不在话下。\par
    3.中国经济的腾飞\par
    大疆无人机公司位于深圳,这里是中国经济飞速发展的一个窗口,基础工业比较发达,因此制造无人机的成本也比较低,因此无人机的费用也就自然而然地降低了。而只有降低无人机的费用,才能真正打开民用无人机的市场。这种条件近二十年间只可能发生在中国,因为发达国家的工业已经退化,其他国家的工业水平又达不到这个标准,大疆在这一方面可以说是占据了地利。\par
    4.汪滔及其导师的研究方向\par
    大疆创新科技有限公司的创始人是香港科技大学的汪滔,汪滔在其导师李泽湘的支持下进行无人机的研究,当时的无人机要想使用必须由用户自己找到自己的组件并下载代码,因此体验不是很好,产品的可靠性也不行。而汪滔受此启发,将目光放在了发展商业成品无人机上,满足了人们的需求,因此成功的将大疆推上了正轨。\par
    \item 要想继续降低消费级无人机的成本,可以从哪些方面入手?\par
    无人机的成本不可能一味地下降,我们需要首先搞清楚它的成本下限在哪里。无人机的电池、飞控和通信模块等都是无人机必不可少的部分,而作为消费级无人机,摄像头和云台亦是不可或缺,这些无法削减的零件使得无人机的价格不能一降再降,若想降低消费级无人机的成本,需要从其他方面入手。\par
    Titania(西班牙卡迪兹)通过其低成本热固性增强项目(MALTA,2020年),正在为无人机(UAV)结构开发一种新的高压釜(OOA)复合制造工艺。据报道,与传统的制造方法相比,该工艺将使无人机零部件的生产成本降低百分之十五。\citep{a6}\par
    无人机作为特殊的飞行器材,它的设计不需要考虑人的生理承受能力,可以充分发挥碳纤维复合材料的性能,考虑更大的设计空间;兼顾无人机的载荷分布和受力情况,博实碳纤维通常采用的是机壳一体化设计,可减少重量百分之三十左右,同时也减少了紧固件零部件的数量,安全性更有保障。并且针对不同使用环境的无人机,可以研究设计出合适的气动外形。\cite{a7}\par
    \item 军用无人机未来发展趋势\par
    向机体小型化、微型化方向发展\par
为了面对不断增强的地面防空火力的威胁,提高自身生存能力,目前无人机正在向隐身化方向发展,许多先进的隐形技术被利用到新型无人机的研制发展上来。美军研制出一种长度和翼展都不超过15厘米的微型无人机,用于特殊条件下的侦察或附在某些物体上搜集视听信息。\par
向机身隐身化方向发展\par
新型无人机将采用最先进的隐身技术。一是采用复合材料、雷达吸波材料和低噪声发动机。美军“捕食者”无人机的机身除了主梁以外,全部采用了石墨合成材料,并对发动机进出气口和卫星通信天线做了特殊设计,其雷达信号特征只有0.1平方米,对雷达、红外和声传感器都有很强的隐身能力。二是采用限制红外反射技术。在无人机表面涂上能吸收红外光的特制漆和在发动机燃料中注入防红外辐射的化学制剂,雷达和目视侦察均难以发现采用这种技术的无人机。三是减少表面缝隙。采用新工艺将无人机的副翼、襟翼等各传动面都制成综合面,进一步减少缝隙,缩小雷达反射面。四是采用充电表面涂层。充电表面涂层主要有抗雷达和目视侦察两种功能。无人机蒙皮由24伏电源充电后,表面即可产生一层能吸收雷达波的保护层。再如美国2002年,波音公司在秘密研制10年后推出的新型隐形无人机“猎鸟”。该机具有极低的雷达反射截面,全新的隐形外观,优秀的隐形特性,达到了隐形的目的。美国空军研究实验室与诺思罗普・格鲁曼公司正在联合研制一种被称为“传感器飞机”的空中无人隐形侦察系统。该机航程3,150~5,560公里,载荷1,820公斤,巡航高度19,800米,巡航时间48小时,装备各种机载传感器系统,包括光电、红外和激光成像系统集成的低、高频雷达,预计将于2015年至2020年期间投入使用。\par

向高空、长航时方向发展\par
续航时间短、飞行高度低的无人机,因侦察监视面积小,不能连续获取信息,往往会造成情报“盲区”,已不能适应现代战争的需要。因此,高空、长航时是未来无人机发展的必然趋势之一。美军在伊拉克战争中使用的高空长航时“全球鹰”无人机,其续航时间在42小时以上,最大飞行高度20,000米,最大飞行距离26,000公里,巡航速度635公里/小时,可从美国本土飞往全球任何地区进行战略和战役侦察。美国国防部先进项目局已与波音公司签订了无人机燃料电池动力系统开发合同,新的燃料电池动力系统能使无人机在空中连续飞行数周,而不是现在的数十小时。\par

向传感器综合化、数传方式多样化、机载设备模块化方向发展\par
英制“小妖精”无人机可根据担负的战场监视、目标指示、电子战等不同任务,分别搭载传感器、激光目标指示器和电子干扰机等各种不同设备。在靶机上装上侦察、电子战设备或各种战斗部,即可将其改装成侦察、电子战无人机或巡航导弹。飞行控制自动化在无人机上安装全球定位系统(GPS)或预先储存飞行路线和飞行高度,无人机即可按预定方案飞行,并随时将图像轨迹发送到地面站。\par
向武器化方向发展\par
随着无人机技术和机载遥感技术,特别是精确制导武器技术的发展,无人机已成为精确制导武器的理想平台。对于主要用于侦察的无人机,出现了一种不同程度武器化的趋势,武器化已成为无人机发展的重要方向之一。美国空军研究实验室认为,无人作战飞机的出现将是战术性空中力量的一场革命。目前,世界各国都把研制无人战斗机作为发展重点,美国更是走在了世界的前列。专家预测,在未来10年内美国纵深攻击战术飞机的三分之一将是无人机。2002年5月,美国波音公司成功进行了X-45A无人战斗机验证机的首次试飞。该机采用武器内挂设计,机身中线两侧各有一个武器舱,可根据作战需要挂载“联合直接攻击弹药”(JDAM)和其他小型精确制导武器。美国空军研究实验室正在研制2010年之后的无人作战飞机一“空间作战飞行器”(SOV)和“未来攻击飞行器”(FSV)等。“未来攻击飞行器”有望于2020至2025年间取代美国空军现役的战略轰炸机。\citep{a8}\par
\end{itemize}


\section{总结}
《计算科学导论》带我了解了计算机历史、现状与未来,尽管目前书中很多内容还一知半解,但不可否认,它拓宽了我的知识领域,增加了我思考问题的维度,现在看不懂也没有什么大问题,我会将这本书好好保存,等到大三大四的时候再拿出来慢慢回味,相信到那时自己一定会有不一样的体会。\par


\section{附录}
\begin{itemize}
    \item 申请Github账户,给出个人网址和个人网站截图
    \item 注册观察者、学习强国、哔哩哔哩APP,给出对应的截图
    \item 注册CSDN、博客园账户,给出个人网址和个人网站截图
    \item 注册小木虫账户,给出个人网址和个人网站截图
\end{itemize}
GitHub网址:https://github.com/siiight/jisuankexuedaolun\par
\begin{figure}[H]
\centering
\includegraphics[scale=0.2]{github}
\caption{github}
\label{fig:universe}
\end{figure}

\begin{figure}[H]
\centering
\includegraphics[scale=0.1]{观察者}
\caption{观察者}
\label{fig:universe}
\end{figure}

\begin{figure}[H]
\centering
\includegraphics[scale=0.1]{学习强国}
\caption{学习强国}
\label{fig:universe}
\end{figure}

\begin{figure}[H]
\centering
\includegraphics[scale=0.1]{B站}
\caption{B站}
\label{fig:universe}
\end{figure}

\begin{figure}[H]
\centering
\includegraphics[scale=0.2]{CSDN}
\caption{CSDN}
\label{fig:universe}
\end{figure}

\begin{figure}[H]
\centering
\includegraphics[scale=0.2]{博客园}
\caption{博客园}
\label{fig:universe}
\end{figure}

\begin{figure}[H]
\centering
\includegraphics[scale=0.2]{小木虫}
\caption{小木虫}
\label{fig:universe}
\end{figure}


\hspace*{\fill} \\


\bibliographystyle{plain}
\bibliography{references}


\end{document}
